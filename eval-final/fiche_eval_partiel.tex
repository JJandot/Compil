\documentclass[12pt,a4paper]{article}
\usepackage[utf8]{inputenc}
\usepackage[french]{babel}
\usepackage{fancyheadings}
\usepackage{color}
\usepackage{graphicx}
\usepackage{epsfig}
\usepackage[top=2cm]{geometry}
%\usepackage{fguill}
\usepackage{tipa}
\newcommand{\gui}[1]{\og{#1}\fg}
\newcommand{\ouf}{\vspace{3mm}}

\newcounter{serie} \setcounter{serie}1
\newcounter{numexo}
\newcommand{\exonum}{Q.\theserie.\addtocounter{numexo}{1}\thenumexo.}

%%%%%%%%%%%%%%%%%%%%%%%%%%%%%%%%%%%%%%%%%%%%%%%%%%%%%%%%%%%%%%%%%%%%%%%%%%%%%

\title{Evaluation du compilateur 2016/2017} 
\author{}
\date{}
%\date{4 janvier 2009}

\begin{document}
\setlength{\parindent}{0cm}

% \maketitle

\lhead{\emph{Université Aix Marseille - L3 Informatique}}
\rhead{\emph{Compilation}}
%\setlength{\headrulewidth}{0.25pt} 
\thispagestyle{fancy}
%%%%%%%%%%%%%%%%%%%%%%%%%%%%%%%%%%%%%%%%%%%%%%%%%%%%%%%%%%%%%%%%%%%%%%%%%%%%%

\begin{center}
 \textbf{\Large{ Evaluation du compilateur 2016/2017}}
\end{center}


NOM~:~~~~~~~~~~~~~~~~~~~~~~~~~~~~~~~~~~~~~~~~~~PRENOM~:\\
NOM~:~~~~~~~~~~~~~~~~~~~~~~~~~~~~~~~~~~~~~~~~~~PRENOM~:\\


%Le sujet de l'évaluation se trouve à l'adresse suivante~:\\
%{\tt http://pageperso.lif.univ-mrs.fr/\~{}alexis.nasr/Ens/Compilation/eval\_2012.html}\\


\subsection*{0 Compilation de l'analyseur}

\begin{tabular}{|l|l|} \hline
Module & ~~~~~~~~~~~~~~~~~~~~~~~~~~~~~Remarques~~~~~~~~~~~~~~~~~~~~~~~~~~~~~\\ \hline
Table des symboles & \\ 
&\\
&\\
& \\\hline
Génération de code & \\ 
&\\
&\\
& \\\hline
\end{tabular}

\subsection*{1 Exécution sur des exemples existants}

\begin{tabular}{|p{2.7cm}|l|l|} \hline
source       &génération & exécution \\ \hline
{\tt affect.l} &~~~~~~~~~~~~~~~~      &~~~~~~~~~~~~~~~~         \\ \hline
{\tt boucle.l} &      &         \\ \hline
{\tt expression.l} &      &         \\ \hline
{\tt max.l} &      &         \\ \hline
{\tt tri.l} &      &         \\ \hline 
\end{tabular}

\subsection*{2 Exécution sur de nouveaux exemples}

\subsubsection*{2.1 Exemples corrects}

\begin{tabular}{|p{3.7cm}|l|l|} \hline
source       &génération & exécution \\ \hline
{\tt procedure\_arg.l} &~~~~~~~~~~~~~~~~ &~~~~~~~~~~~~~~~~        \\ \hline
{\tt procedure.l} &            &         \\ \hline
{\tt procedure\_retour.l}       &      &         \\ \hline
{\tt procedure\_varloc.l}       &      &         \\ \hline
{\tt fibo.l} &            &         \\ \hline
{\tt associativite.l} &            &         \\ \hline
{\tt pgcd.l} &      &               \\ \hline
{\tt condexp.l} &      &               \\ \hline
{\tt imbrique.l} &      &               \\ \hline
\end{tabular}

%\begin{tabular}{|l|l|l|} \hline
%fichier de test & compilation & exécution \\ \hline
%{\tt eval4.l} &     &      &         &     &     \\ \hline
%{\tt eval5.l} &     &      &         &     &     \\ \hline
%{\tt eval6.l} &     &      &         &     &     \\ \hline
%\end{tabular}

\subsubsection*{2.1 Exemples incorrects}

\begin{tabular}{|p{3.4cm}|l|} \hline
source       & génération \\ \hline
{\tt semantique1.l} &~~~~~~~~~~~~~~~~  \\ \hline
{\tt semantique2.l} &~~~~~~~~~~~~~~~~  \\ \hline
{\tt semantique3.l} &~~~~~~~~~~~~~~~~  \\ \hline
{\tt semantique4.l} &~~~~~~~~~~~~~~~~  \\ \hline
{\tt semantique5.l} &~~~~~~~~~~~~~~~~  \\ \hline
{\tt semantique6.l} &~~~~~~~~~~~~~~~~  \\ \hline
\end{tabular}

%\section{Ajout du type {\tt reel}}
\subsection*{3.1 Implémentation de l'instruction {\tt incr}}

\begin{tabular}{|p{3cm}|p{12cm}|} \hline
& \\
programmation & \\ 
& \\\hline
& \\
compilation & \\ 
& \\\hline
\end{tabular}

\subsection*{4 Tests}

\begin{tabular}{|p{3.4cm}|l|l|l|l|} \hline
source       & 1 & 2 & 3 & 4 \\ \hline
{\tt incr1.l} &~~~~~~~~~~~~~~~~ &~~~~~~~~~~~~~~~~      &~~~~~~~~~~~~~~~~  &~~~~~~~~~~~~~~~~         \\ \hline
{\tt incr2.l} &~~~~~~~~~~~~~~~~ &~~~~~~~~~~~~~~~~      &~~~~~~~~~~~~~~~~  &~~~~~~~~~~~~~~~~         \\ \hline
\end{tabular}


\end{document}
